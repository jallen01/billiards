We will now use the tiling representation that we have developed to discover properties of collision sequences. The first simple property is that collision sequences are periodic when the initial conditions are rational numbers. Now that we can define a billiard ball as a line, instead of giving initial conditions $\bvec{x}_0$ and $\bvec{v}_0$, we can give the slope $m$ and the $y$-intercept $y_0$ of the complete trajectory's line. The formalized theorem is then:

\begin{theorem}
  \label{theorem:periodicity}
  There exists a $k \in \mathrm{N}$ such that $mk + y_0 \equiv y_0 \pmod{2}$ if and only if $m \in \mathrm{Q}$.
\end{theorem}
\begin{proof}
\end{proof}

Theorem \ref{theorem:periodicity} shows that if $m \in \mathrm{Q}$, then the billiard ball will eventually return to it's original position $\bvec{x}_0$ with its original velocity $\bvec{v}_0$. Seeing why this is true is just a matter of using the tiling representation. We note that every second square in either the $x$ or $y$ direction is the same (because of the transitivity of reflection). Therefore, every second square will have a trajectory that exactly corresponds to the trajectory in the original square.
