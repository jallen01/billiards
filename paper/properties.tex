We will now use the tiling representation that we have developed to discover properties of collision sequences. The first simple property is that collision sequences are periodic when the initial conditions are rational numbers. Now that we can define a billiard ball trajectory as a line, instead of giving initial conditions $\bvec{x}_0$ and $\bvec{v}_0$, we can give the slope $m$ and the $y$-intercept $y_0$ of the complete trajectory's line. The formalized theorem is then:

\begin{theorem}
  \label{theorem:periodicity}
  There exists a $k \in \mathrm{N}$ such that $y_0 \equiv mk + y_0 \pmod{2}$ if and only if $m \in \mathrm{Q}$.
\end{theorem}
\begin{proof}
  Let us first show that if $m \in \mathrm{Q}$, then there exists a $k \in \mathrm{N}$ such that $y_0 \equiv mk + y_0 \pmod{2}$. We simply need to show that $0 \equiv mk \pmod{2}$ if $m \in \mathrm{Q}$. However, since we know $m \in \mathrm{Q}$, we can decompose it as follows $m = p/q$ where $p,q \in \mathrm{Z}$. Thus, we have $mk \pmod{2} \equiv \frac{pk}{q} \pmod{2}$. Now we can choose:
  \begin{eqnarray}
    k = \left\{\begin{array}{l c}
      q & \textrm{if } p \pmod{2} \equiv 0 \\
      2q & \textrm{if } p \pmod{2} \equiv 1
    \end{array}\right.
  \end{eqnarray}
  In this way, we see that $0 \equiv \frac{pk}{q} \pmod{2}$, which proves the first half of the theorem.

  Now let us show that if there exists a in $k \in \mathrm{N}$ such that $y_0 \equiv mk + y_0 \pmod{2}$, then $m \in \mathrm{Q}$. If such a $k$ exists, then we must have $0 \equiv mk \pmod{2}$, which means that $mk = 2q$ for some $q \in \mathrm{Z}$. This means $m = \frac{2q}{k}$. Now it is clear that $m \in \mathrm{Q}$ because both its numerator and denominator are integers.
\end{proof}

Theorem \ref{theorem:periodicity} shows that if $m \in \mathrm{Q}$, then the billiard ball will eventually return to it's original position $\bvec{x}_0$ with its original velocity $\bvec{v}_0$. Seeing why this is true is just a matter of using the tiling representation. We note that every second square in either the $x$ or $y$ direction is the same (because of the transitivity of reflection). Therefore, every second square will have a trajectory that exactly corresponds to the trajectory in the original square.

Thus, if $y_0 \equiv mk + y_0 \pmod{2}$, then the $y$-intercept from one of the secondary squares is the same as the $y$-intercept from the original square. Since we know that the secondary squares have the same trajectories as in the original square, we see that the trajectory will have returned to its original position (since the velocity is the same). Thus, if $y_0 \equiv mk + y_0 \pmod{2}$, then we know that the billiard ball will return to its original position with its original velocity. Theorem \ref{theorem:periodicity} also shows that if $m$ is irrational, then the billiard ball will never return to its original position and velocity.

We can also examine consecutive occurences of $v$ and $h$. For example, can we have consecutive occurences of both $v$ and $h$ like in the sequence $hhvvhh$? In fact, we cannot as theorem \ref{theorem:consecutive-collisions} shows.

\begin{theorem}
  \label{theorem:consecutive-collisions}
  A valid collision sequence cannot have both consecutive occurrences of $v$ and consecutive occurences of $h$.
\end{theorem}
\begin{proof}
  We have already shown in theorem \ref{theorem:straight-line} that the combined trajectory of a billiard ball must be a ray. This ray must lie on some line with some slope $m \in \mathrm{R}$ or with undefined slope (when the line is vertical). When the line is vertical, it is clear that the theorem holds, because only $v$ collisions occur. 

  There are two cases left: $|m| < 1$ or $|m| \geq 1$. If $|m| \geq 1$, then the number of $v$-collisions between the $x$th and the $x+1$st $h$-collision will be $\lfloor (m(x+1) + y_0) - (mx + y_0) \rfloor = \lfloor m \rfloor \geq 1$. Thus, we see that for any $x \in \mathrm{N}$, we must have at least 1 $v$-collision between the $x$th and $x+1$st $h$-collisions, which proves the theorem for $|m| \geq 1$.

  If $|m| < 1$, then the number of $h$-collisions between the $y$th and $y+1$st $v$-collisions will be $\lfloor ((y + 1) - y_0)/m - (y - y_0)/m \rfloor = \lfloor 1/m \rfloor > 1$. This shows that there will be at least 1 $h$-collision between the $y$th and $y+1$st $v$-collisions for all $y \in \mathrm{N}$. This completes the theorem.
\end{proof}

We therefore see that if $v$ occurs consecutively in a collision sequence, then $h$ cannot occur consecutively and vice versa. For example, the sequence $hvvhh$ has two consecutive occurrences of $v$ and two consecutive occurrences of $h$, so it cannot be a valid collision sequence. However, the sequence $hvvhvvvh$ does not have consecutive occurrences of $h$, so theorem \ref{theorem:consecutive-collisions} does not rule it out as a valid collision sequence.

Theorem \ref{theorem:consecutive-collisions} allows us to make a simplification for our collision sequences. Since one of either $v$ or $h$ must occur non-consecutively, we can abitrarily assign $v$ to be the side that occurs non-consecutively by rotating the billiard table and creating the opposite collision sequence. For example, the sequence $vhhhvhhhv$ is the same as the sequence $hvvvhvvvh$ when one rotates the billiard table by $\pi/2$. Therefore, from now on, we can confine all our sequences to have only non-consecutive occurrences of $v$ (i.e. sequences like $hvvvhvvvh$ become $vhhhvhhhv$) without loss of generality.
