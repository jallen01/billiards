%
%  beamer_template
%
%  Created by Christopher Schommer-Pries on 2012-02-16.
%  Copyright (c) 2012. All rights reserved.
%


\documentclass{beamer}		%% The Beamer document class formats for slides. 

\usepackage{presentation}

\title{A Title Should go here}
\author{A. Sheep, A. Goat, and A. Cow}
\institute[MIT]{Massachusetts Institute of Technology}
%\date{June $6^\text{th}$, 2011}

\begin{document}

\frame{
\titlepage  % This makes the title. 

}

\frame{
This is a slide!
}


\frame{
\frametitle{This slide has a title!}

\pause
Wow!
\pause

You can also make lists...
\begin{itemize}
	\item with an item \pause
	\item and another item.
\end{itemize}

}


\frame{
You can typeset equations just as in latex:
\[
\int_0^1 x\, dx = 1/2,
\]
or
$$
\sum_{i = 1}^\infty i = -\frac{1}{12}
$$
or
\begin{equation*}
1 - 1 + 1 - \cdots = \frac{1}{2}.
\end{equation*}

}


\frame{
If you want a number for an equation, do it like this:
\begin{equation}\label{eq:first-equation}
\lim_{n \to \infty}\, \sum_{k = 1}^n \frac{1}{k^2} = \frac{\pi}{6}.
\end{equation} %
This can then be referred to as \eqref{eq:first-equation}, which is much easier than
keeping track of numbers by hand. To group several equations, aligning on the $=$ sign, do
it like this:
\begin{align*}
x_1 + 2x_2 + 3x_3 &= 7 \\ y &= mx + c \\ &= 4x - 9.
\end{align*}
}

\frame{
\begin{theorem}[my great theorem]
	This theorem proves everything.
\end{theorem}

\begin{example}
	For example it proves \alert{this}! % "alert" makes it red
\end{example}

}



\end{document}

